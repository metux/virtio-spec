\section{General Purpose IO Device}\label{sec:Device Types / General Purpose IO}

The virtio gpio device is a general purpose IO device that supports a variable
number of named IO lines that may be switched either as input or output and
in logical level 0 or 1.

\subsection{Device ID}\label{sec:Device Types / General Purpose IO / Device ID}
  41

\subsection{Version}\label{sec:Device Types / General Purpose IO / Version}
  1

\subsection{Device configuration layout}\label{sec:Device Types / General Purpose IO / Device configuration layout}

General purpose IO configuration uses the following layout structure:

\begin{lstlisting}
struct virtio_gpio_config {
    __u8    version;
    __u8    reserved0;
    __u16   num_gpios;
    __u32   names_size;
    __u8    reserved1[24];
    __u8    name[32];
    __u8    line_names[];
};
\end{lstlisting}

\begin{itemize}
    \item for \field{version} field currently only value 1 supported.
    \item the \field{line names block} holds a stream of zero-terminated strings,
        containing the individual line names in ASCII. line names must unique.
    \item unspecified fields are reserved for future use and should be zero.
    \item future versions may extend this configuration space by additional fields.
\end{itemize}

\subsection{Virtqueues}\label{sec:Device Types / General Purpose IO / Virtqueues}
\begin{description}
\item[0] rx (device to CPU)
\item[1] tx (CPU to device)
\end{description}

The virtqueues transport messages of the type struct virtio_gpio_msg from device to CPU or CPU to device.

\subsubsection{Virtqueues}\label{sec:Device Types / General Purpose IO / Virtqueues / Message format}

The queues transport messages of the struct virtio_gpio_msg:

\begin{lstlisting}
struct virtio_gpio_msg {
    __le16  type;
    __le16  pin;
    __le32  value;
};
\end{lstlisting}

\subsubsection{Message types}\label{sec:Device Types / General Purpose IO / Virtqueues / Message types}

\begin{lstlisting}
/* messages types: driver -> device */
#define VIRTIO_GPIO_MSG_CPU_REQUEST             0x0001
#define VIRTIO_GPIO_MSG_CPU_DIRECTION_INPUT     0x0002
#define VIRTIO_GPIO_MSG_CPU_DIRECTION_OUTPUT    0x0003
#define VIRTIO_GPIO_MSG_CPU_GET_DIRECTION       0x0004
#define VIRTIO_GPIO_MSG_CPU_GET_LEVEL           0x0005
#define VIRTIO_GPIO_MSG_CPU_SET_LEVEL           0x0006

/* message types: device -> driver */
#define VIRTIO_GPIO_MSG_DEVICE_LEVEL            0x0011

/* reply mask: device sets this bit on replies (along with request's message type)
#define VIRTIO_GPIO_MSG_REPLY                   0x8000
\end{lstlisting}

\devicenormative{\subsubsection}{Virtqueues}{Device Types / General Purpose IO / Virtqueues}

The device MUST read from the tx queue and write to rx queue.

The device MUST NOT write to the tx queue.

\drivernormative{\subsubsection}{Virtqueues}{Device Types / General Purpose IO / Virtqueues}

The device MUST read from the rx queue and write to tx queue.

The device MUST NOT write to the rx queue.

\subsection{Data flow}\label{sec:Device Types / General Purpose IO / Data flow}

\begin{itemize}
    \item all operations, except \field{VIRTIO_GPIO_MSG_DEVICE_LEVEL}, are initiated by CPU (tx queue)
    \item device replies with the orinal \field{type} value OR'ed with \field{VIRTIO_GPIO_MSG_REPLY} (rx queue)
    \item requests are processed and replied in the they had been sent
    \item async notifications by the device may be interleaved with request responses
    \item VIRTIO_GPIO_MSG_DEVICE_LEVEL is only sent asynchronously from device to CPU
    \item in replies, a negative \field{value} field denotes an Unix-style / POSIX errno code
    \item the actual error values \textit{(despite being negative or not)} is only of informational
          nature -- a device or vm host \textit{may} report more detailed error cause but is not required to.
    \item valid direction values are: 0 = output, 1 = input
    \item valid line level values are: 0 = inactive, 1 = active
    \item CPU should not send messages with unspecified \field{type} value
    \item CPU should ignore ignore messages with unspecified \field{type} value
\end{itemize}


\subsubsection{VIRTIO_GPIO_MSG_CPU_REQUEST}\label{sec:Device Types / General Purpose IO / Data flow / VIRTIO-GPIO-MSG-CPU-REQUEST}

Notify the device that given line number is going to be used.

\begin{tabular}{ll}
    \hline
    \textbf{request:} & \\
    \hline
    \field{line}  field: & logical line number \\
    \field{value} field: & unused (should be zero) \\
    \textbf{reply:} & \\
    \hline
    \field{value} field: & POSIX errno code (0 = success, non-zero = error) \\
    \hline
\end{tabular}

\subsubsection{VIRTIO_GPIO_MSG_CPU_DIRECTION_INPUT}\label{sec:Device Types / General Purpose IO / Data flow / VIRTIO-GPIO-MSG-CPU-DIRECTION-INPUT}

Set line line direction to input.

\begin{tabular}{ll}
    \hline
    \textbf{request:} \\
    \hline
    \field{line}  field: & logical line number \\
    \field{value} field: & unused (should be zero) \\
    \hline
    \textbf{reply:} & \\
    \hline
    \field{value} field: & POSIX errno code (0 = success, non-zero = error) \\
    \hline
\end{tabular}

\subsubsection{VIRTIO_GPIO_MSG_CPU_DIRECTION_OUTPUT}\label{sec:Device Types / General Purpose IO / Data flow / VIRTIO-GPIO-MSG-CPU-DIRECTION-OUTPUT}

Set line direction to output and given line level.

\begin{tabular}{ll}
    \hline
    \textbf{request:} \\
    \hline
    \field{line}  field: & logical line number \\
    \field{value} field: & output level (0=inactive, 1=active) \\
    \hline
    \textbf{reply:} & \\
    \hline
    \field{value} field: & POSIX errno code (0 = success, non-zero = error) \\
    \hline
\end{tabular}

\subsubsection{VIRTIO_GPIO_MSG_CPU_GET_DIRECTION}\label{sec:Device Types / General Purpose IO / Data flow / VIRTIO-GPIO-MSG-CPU-GET-DIRECTION}

Retrieve line direction.

\begin{tabular}{ll}
    \hline
    \textbf{request:} & \\
    \hline
    \field{line}  field: & logical line number \\
    \field{value} field: & unused (should be zero) \\
    \hline
    \textbf{reply:} & \\
    \hline
    \field{value} field: & direction (0=output, 1=input) or POSIX errno code \\
    \hline
\end{tabular}

\subsubsection{VIRTIO_GPIO_MSG_CPU_GET_LEVEL}\label{sec:Device Types / General Purpose IO / Data flow / VIRTIO-GPIO-MSG-CPU-GET-LEVEL}

Retrieve line level.

\begin{tabular}{ll}
    \hline
    \textbf{request:} & \\
    \hline
    \field{line}  field: & logical line number \\
    \field{value} field: & unused (should be zero) \\
    \hline
    \textbf{reply:} & \\
    \hline
    \field{value} field: & line level (0=inactive, 1=active) or errno code \\
    \hline
\end{tabular}

\subsubsection{VIRTIO_GPIO_MSG_CPU_SET_LEVEL}\label{sec:Device Types / General Purpose IO / Data flow / VIRTIO-GPIO-MSG-CPU-SET-LEVEL}

Set line level (output only)

\begin{tabular}{ll}
    \hline
    \textbf{request:} & \\
    \hline
    \field{line}  field: & logical line number \\
    \field{value} field: & line level (0=inactive, 1=active) \\
    \hline
    \textbf{reply:} & \\
    \hline
    \field{value} field: & new line level or (negative) POSIX errno code \\
    \hline
\end{tabular}

\subsubsection{VIRTIO_GPIO_MSG_DEVICE_LEVEL}\label{sec:Device Types / General Purpose IO / Data flow / VIRTIO-GPIO-MSG-DEVICE-LEVEL}

Async notification from device to CPU: line level changed

\begin{tabular}{ll}
    \hline
    \textbf{request:} & \\
    \hline
    \field{line}  field: & logical line number \\
    \field{value} field: & unused (should be zero) \\
    \hline
    \textbf{reply:} & \\
    \field{value} field: & line level (0=inactive, 1=active) \\
    \hline
\end{tabular}

\devicenormative{\subsubsection}{Data flow}{Device Types / General Purpose IO / Data flow}

The device MUST reply to all driver requests in they had been sent.

The device MUST copy the \field{line} field from the request to its reply.

Except for async notification, the device MUST reply the orignal message type, but with the highest bit set
(or'ed with VIRTIO_GPIO_MSG_REPLY)

In case of error the device MUST fill an negative value into the \field{value} field, it SHOULD use
an fitting POSIX / Unix errno value when applicable.

On switching to output, the device SHOULD set internal output level before switching the line to output.

The device SHOULD send VIRTIO_GPIO_MSG_DEVICE_LEVEL message for a particular line when it is in input
direction and line level changes.

\drivernormative{\subsubsection}{Data flow}{Device Types / General Purpose IO / Data flow}

The driver MUST NOT send VIRTIO_GPIO_MSG_DEVICE_LEVEL and MUST NOT set the highest bit in the message type.

The driver MUST NOT send messages with types not defined in this specification.

\subsection{Future versions}\label{sec:Device Types / General Purpose IO / Future versions}

\begin{itemize}
    \item future versions must increment the ``version`` value
    \item the basic data structures (config space, message format) should remain
          backwards compatible, but may increased in size or use reserved fields
    \item device needs to support commands in older versions
    \item CPU should not send commands of newer versions that the device doesn't support
\end{itemize}

\subsection{Feature bits}\label{sec:Device Types / General Purpose IO / Feature bits}

There are currently no feature bits defined for this device, but may be added in future versions.
